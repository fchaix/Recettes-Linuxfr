\documentclass[%
	a4paper,
	10pt,
	linktocpage=true,
	oneside,
	DIV=calc, 
	% DIV règle les proportions corps de texte-marges.
	% Calc c’est pour qu’il calcule automatiquement la taille du corps du
	% texte en fonction de la taille de la police utilisée.
	]{scrreprt}
% scrreprt est la classe qui remplace la classe « report » de LaTeX, avec la
% suite KOMA-script. En gros c’est la même chose, mais avec des choses mieux
% pour la typographie des langues européennes, la gestion du nombre de
% caractères par ligne, etc.
% Des explications ici : http://lesfichesabebert.fr/index.php/Koma/Koma-Script

\usepackage[francais]{babel}
\usepackage{fontspec}
    \setmainfont[%
	Numbers=OldStyle,
	Ligatures={
		TeX,
		Common,
		Rare,
		Historical%
		% Oui, je suis un petit peu vieux jeu…
		}
	]{Linux Libertine O}
    \setsansfont{Linux Biolinum O}
\KOMAoptions{DIV=last} 
% La définition de DIV (Cf. options de \documentclass), dans le cas de calc,
% doit se déclarer après la déclaration de la police.

\usepackage[% Propriétés du PDF, et options des liens
    hidelinks=true
    colorlinks=false
    pdfauthor={François Chaix},
    pdftitle={Recettes de libristes — Da Boustifaille French Page},
    pdfdisplaydoctitle=true, % Display document title instead of filename in title bar
    pdfsubject={plop},
    pdfkeywords={plop},
    pdfproducer={LuaTeX, avec le package hyperref},
    pdfcreator={LuaTex},
    linktocpage=false,
    pdfinfo={pouet ?},
    pdflang={fr-FR},
    unicode=true,
    verbose=true
    ]{hyperref}

\usepackage{lipsum} % Temporaire, pour les tests

\title{Mon super titre que je trouverais plus tard}
\author{François \textsc{Chaix}}


\newcommand{\lien}%
	[1]% 1 argument
	{% Effets de la commande
	\href{#1}{Lien sur linuxfr}%
	}

% La commande \lien insère le lien mis dans le premier argument, avec le texte
% "Lien sur linuxfr".

% ToDo : Indéxer les liens
 % Ça ne marche pas de mettre un *
\newcommand{\moule}%
	[2]% 2 arguments
	{%
		\href{https://linuxfr.org/users/#2}{#1}%
	}

% La fonction \moule affiche le nom de la moule sur linuxfr, comme texte d'un
% lien vers son profil.

% ToDo : Indéxer les moules.

\newenvironment{remarque}
	{% Ce qu'il se passe en début d'environement
		\begin{quote}
		\small%
	}
	{% Ce qu'il se passe en fin d'environement
		\normalsize
		\end{quote}%
	}

% L'envirmnement remarque n'est autre qu'un environement de citation avec une
% police réglée en small

% Cet environement est un environement de type liste. À utiliser comme suit :
% \begin{ingredients}
% \item truc
% \end{ingredients}

\newenvironment{ingredients}%
{ \begin{list}%
	{$\bullet$}% Cet argument permet de définir le symbole qui sera
	%associé à chaque élément de la liste.
	{% Mise en page de la liste
		\setlength{\labelwidth}{30pt}%
		\setlength{\leftmargin}{35pt}%
		\setlength{\itemsep}{\parsep}}}%
{ \end{list} }
% Cet environement est un environement de type liste. À utiliser comme suit :
% \begin{instructions}
% \item truc
% \end{instructions}

\newenvironment{instructions}%
{ \begin{list}%
	{$\spadesuit$}% Cet argument permet de définir le symbole qui sera
	%associé à chaque élément de la liste.
	{% Mise en page de la liste
		\setlength{\labelwidth}{30pt}%
		\setlength{\leftmargin}{35pt}%
		\setlength{\itemsep}{\parsep}}}%
{ \end{list} }


