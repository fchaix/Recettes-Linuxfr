\section{Tarte poires amandes chocolat}
	\moule{Pierre Tramonson}{fredx}
	\lien{https://linuxfr.org/users/fredx/journaux/recette-de-tarte-poires-amandes-chocolat}
	\Date{05/07/13}

L'été est enfin arrivé, on a envie de se vautrer dans une chaise longue avec
un jus de fruit frais, c'est le moment idéal pour se faire plaisir avec un bon
dessert.

Je vous propose une recette simple de tarte poires amandes chocolat.

\begin{ingredients}
	\item 1 rouleau de pâte brisée (mais vous pouvez également prendre une
	pâte feuilletée, ou faire vous même la pâte, je fais au plus simple)
	\item 4 grosses poires mures
	\item 3 \oe{}ufs
	\item 300\,g de sucre en poudre
	\item 125\,g de poudre d'amande
	\item 175g de beurre
	\item du chocolat à dessert (moins de 100g)
\end{ingredients}

\subsection*{Préparation des poires}

Pelez les poires et coupez les en deux, évidez les de leurs pépins. Les placer
dans une casserole avec $\frac{1}{2}$ litre d'eau et 150\,g de sucre. Porter à ébullition
et laisser cuire 15\,min (s'il manque de l'eau, rajoutez en, il faut qu'il reste
du sirop). Vérifiez la cuisson en piquant les poires avec un couteau, qui doit
s'enfoncer facilement.

	\begin{remarque}

		vous pouvez remplacer l'eau par du vin rouge (pas trop puissant de
		préférence) et quelques cuillères à café de crème de cassis. Le gout
		sera complètement différent. Il n'est pas utile de prendre du rouge
		bien frais, même si votre pote chauve vous le conseille.

	\end{remarque}

	\begin{remarque}

		Si vous êtes une grosse feignasse, vos pouvez prendre directement des
		poires au sirop en boite.

	\end{remarque}

\subsection*{Préparation de la tarte}

\begin{instructions}
	
	\item Préchauffez le four à 210°C (th 7).
	\item Faites fondre le beurre dans une casserole. Mélangez le beurre et
	150\,g de sucre dans un saladier, ajoutez les œufs en continuant à remuer,
	puis la poudre d'amandes.
	\item Beurrez le moule et déroulez-y la pâte brisée.
	\item Étalez la préparation sur la pâte. Répartissez les demi-poires sur
	cette préparation (face bombée vers le haut, évidemment). Si jamais vous
	trouvez que vous n'avez pas assez de poires : vous pouvez couper de fines
	tranches d'une poire bien mure et les intercaler entre les demi-poires
	(les tranches cuiront directement au four). C'est pas mal pour compléter
	une boite de poires au sirop (y en a jamais assez dans ce genre de
	boites).
	\item Saupoudrez l'ensemble de petites pépites de chocolat; soit des
	carrés cassés en morceaux, soit des pépites de chocolat achetées pour
	dessert.

\end{instructions}

\subsection*{Cuisson et dressage}

\begin{instructions}

	\item Laissez cuire 20min à 210°C puis poursuivez 30-35 min (ça dépend du
	four) à 180°C. Quand c'est bien doré, c'est que c'est bon.
	\item Servez la tarte tiède nature, ou avec de la glace vanille, ou de la
	crème fraiche battue (et un peu de sirop de cassis si vous avez mis de la
	crème de cassis dans le sirop au vin rouge).
	\item Vous pouvez également finir la bouteille de rouge avec.

\end{instructions}

Reposez vous sur votre chaise longue, tel l’éléphant de mer sur sa plage, le
ventre plein et les lèvres couvertes de jus de poire et de chocolat fondu.
