\section{Boulangerie autohébergée}
	\moule{devnewton}{devnewton}
	\lien{https://linuxfr.org/users/devnewton/journaux/une-recette-pour-auto-heberger-sa-boulangerie}
	\Date{05/06/13}

Peu de gens le savent, mais il est parfaitement possible de faire du bon pain
à domicile sans machine.

Voici la recette que j'utilise depuis des années.

Les ingrédients et matériels nécessaires sont:

\begin{ingredients}
	\item 500\,g de farine.
	\item un sachet de levure boulangère.
	\item 1,5 cuillère à café de sel.
	\item 300\,g d'eau tiède.
	\item un grand récipient.
	\item une cuillère en bois.
	\item du papier cuisson.
	\item un four.
\end{ingredients}

La fabrication se fait en sept étapes:
\begin{instructions}

	\item Commencez par mélanger la farine, la levure et le sel dans le
	récipient.

	\item Ajoutez l'eau et remuez avec la cuillère en bois jusqu'à obtenir une
	pâte uniforme.

	\item Mettre un torchon sur le récipient et laissez reposer pendant 30
	minutes dans l'endroit le plus chaud de votre demeure.

	\item Allonger le papier cuisson sur la plaque du four et saupoudrer de
	farine.

	\item Mettre la pâte dessus et lui donner une forme allongée ou ronde
	selon votre envie.

	\item Poser un torchon dessus et attendre 45 minutes.

	\item Enlever le torchon, arroser d'un peu d'eau et mettre au four à 200°
	pendant 30 minutes.

\end{instructions}

Vous obtiendrez ainsi un pain meilleur que dans la plupart des boulangeries parisiennes qui peut être consommé pendant 4 à 5 jours.

\begin{remarque}

Il est possible de varier la recette en mixant la farine de blé avec d'autres.
La farine de châtaigne par exemple donne un pain excellent pour accompagner le
fromage et la confiture.

\end{remarque}

\begin{attention}

certaines personnes considèrent que l'auto-hébergement de boulangerie est un
truc de bobo/hipster qui ruine les commerces. Si vous les recevez à dîner,
n'oublier pas de leur prendre une baguette industrielle pour respecter leurs
convictions.

\end{attention}