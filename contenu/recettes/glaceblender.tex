\section{La glace au blender}
	\moule{MrLapinot}{mrlapinot}
	\lien{https://linuxfr.org/users/mrlapinot/journaux/la-glace-au-blender}
	\Date{16/06/13}

En ces temps de chaleur estivale, une recette simple et efficace : la glace au
\textit{blender} (certains préféreront dire « mixeur », c’est plus français
mais pas exactement la même chose).

Il vous faut :
\begin{ingredients}
	\item 3 minutes,
	\item un bon blender (Philips HR2094 chez moi, mais n’importe lequel
	conviendra sûrement tant qu’il est capable de piler de la glace),
	\item des fruits surgelés (framboises et mangue par exemple),
	\item autant de lait que de fruits,
	\item un peu de jus de fruit à votre convenance (jus d’orange par exemple),
	\item ainsi que quelques épices (vanille, cannelle, etc. selon les goûts).
\end{ingredients}

\begin{instructions}

	\item Mettez tous les ingrédients dans le blender, 30 secondes en mode
	«glace pilée», puis une minute en mode «normal» pour homogénéiser. Vous
	obtenez une délicieuse glace maison à consommer immédiatement.

\end{instructions}

Bon appétit !