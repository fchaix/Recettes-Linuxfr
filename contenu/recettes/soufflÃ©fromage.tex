\section{Soufflé au fromage}
	\moule{Pierre Tramonson}{fredx}
	\lien{https://linuxfr.org/users/fredx/journaux/recette-de-souffle-au-fromage}
	\Date{04/06/13}

Avec ce temps hivernal, c'est le moment de faire des plats simples, roboratifs
et conviviaux. Un truc avec du fromage donc. Et pour faire plus évolué que la
raclette, pourquoi ne pas se risquer sur un soufflé au fromage ?  Une recette
simple (vraiment), rapide, sans chichis, qui présente en plus l'avantage de
nous rappeler comment faire une béchamel.

\begin{ingredients}

	\item 125\,g de gruyère rapé
	\item 100\,g de farine
	\item 50\,cl de lait demi-écrémé
	\item 80\,g de beurre
	\item 6 œufs
	\item Noix de muscade, sel, poivre

\end{ingredients}

\subsection*{Préparation de la béchamel}

\begin{instructions}

	\item Dans une casserole, faites fondre le beurre avec la farine.
	\item Salez et poivrez.
    \item Remuez énergiquement sur feu assez vif et versez progressivement le
    lait tout en continuant à remuer (idéalement avec un fouet ou une spatule
    en bois). Montez ainsi une béchamel.
	\item Laissez la cuire quelques minutes pour qu’elle soit épaisse et bien
	lisse.

\end{instructions}

\subsection*{Préparation du soufflé}

\begin{instructions}

	\item Faites préchauffer le four Thermostat 6 (180°C).
	\item Sortez la béchamel du feu. Lorsque la béchamel est tiède, ajoutez le
	fromage, les œufs et la noix de muscade. Mélangez bien.
	\item Battez les blancs en neige ferme (vive le robot !).
	\item Incorporez-les à la béchamel à l’aide d’une spatule de manière à ce
	que le mélange soit bien homogène, sans y aller trop énergiquement pour
	ne pas écraser la neige.

\end{instructions}

\subsection*{Cuisson du soufflé}

\begin{instructions}

	\item Beurrez un moule à soufflé (bords hauts) et remplissez-le de la
	préparation, puis enfournez. Attention un soufflé double de taille,
	choisissez votre moule en fonction !
	\item  Au bout de 10 min, montez la température à th 7 (210°C). Laissez
	cuire encore 35 min (soit 45 en tout). Vous pouvez vérifier la cuisson en
	plantant un couteau dans le soufflé et constater que le cœur n’est plus
	liquide.
	\item Dégustez immédiatement, avec une salade verte par exemple.

\end{instructions}

\begin{figure}[h]
	\begin{center}
		\includegraphics[width=5cm]{contenu/images/soufflé.jpg}
	\end{center}
	\caption{Soufflé au fromage. Photo de Pierre \textsc{Tramonson}}
	\label{fig:souffle}
\end{figure}